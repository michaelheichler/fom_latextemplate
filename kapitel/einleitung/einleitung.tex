% Beispielhafte Introduction

\section{Introduction}\label{intro}
Einleitungskapitel. Kapitel Überschriften orientieren sich an der englischen Vorgabe.

Via 
\verb!\autocite[links vom Zitat][rechts vom Zitat]{Angabe des Bibliothekseintrages}! 
kann eine Zitierung hinzugefügt werden.\autocite[links][rechts]{LeCun2015}

Die Anführungszeichen werden gesondert dargestellt mit \verb!\glqq! und \verb!\grqq! und sieht wie folgt aus: \glqq Zitat\grqq.

Eine Kapitelreferenz (im selben Dokument) kann mit \verb!\ref{put here label from chapter}! eingefügt werden und sieht wie folgt aus: \ref{intro}. Es empfiehlt sich dies etwas erweitert manuell zu schmücken: Kapitel \ref{intro}. 

\subsection{Context}

% neue Zeile kann mit \newpage (s. unten) eingefügt werden
\newpage
\subsection{Problem statement}\label{quest} % Das hinzufügen des \label Kürzels an den Kapitelnamen ermöglicht eine Referenzierung innerhalb des Dokumentes



\subsection{Aim and scope}


\subsection{Significance}


\subsection{Overview}
